\documentclass[12pt]{article}

\usepackage{amssymb,amsthm,amsmath,fullpage}

\newtheorem{definition}{Definition}
\newtheorem{proposition}[definition]{Proposition}
\newtheorem{corollary}[definition]{Corollary}

\newcommand{\R}{\mathbb{R}}
\newcommand{\grad}{\nabla}
\newcommand{\norm}[1]{\left\|#1\right\|}

\title{Smooth Optimization}

\author{D\'avid P\'al}

\begin{document}

\maketitle

We denote by $\norm{\cdot}$ the Euclidean norm on $\R^d$. That is, for a vector $x = (x_1, x_2, \dots, x_d) \in \R^d$,
we define $\norm{x} = \sqrt{\sum_{i=1}^d x_i^2}$. We overload the notation and for a matrix $M$
we denote by $\norm{M}$ the spectral norm of $M$. That is,
$$
\norm{M} = \sup_{\substack{x \in \R^d \\ x \neq 0}} \frac{\norm{Mx}}{\norm{x}} \; .
$$

\begin{definition}[Smoothness]
Let $\eta > 0$. A differentiable function $f:\R^d \to \R$ is called \emph{$\eta$-smooth} if
\begin{equation}
\label{equation:smoothness-definition}
\forall x,y \in \R^d \qquad \norm{\grad f(x) - \grad f(x)} \le \eta \norm{x - y} \; .
\end{equation}
\end{definition}

\begin{proposition}
If $f:\R^d \to \R$ is a $\eta$-smooth function then
\begin{equation}
\label{equation:smoothness-proposition-1}
\forall x,y \in \R^d \qquad \left| f(x) - f(y) - \langle \grad f(y), x - y \rangle \right| \le \frac{\eta}{2} \norm{x - y}^2 \; .
\end{equation}
\end{proposition}

\begin{proof}
We reduce the problem to a one-dimensional one. Fix $x,y \in \R^d$
and define a scalar function $g:\R \to \R$ as $g(t) = f(y + t (x - y))$. Note
that $g(0) = f(y)$, $g(1) = f(x)$ and $g'(t) = \langle \grad f(y + t(x - y)), x -
y \rangle$.

It easy to see that $g$ is $(\eta \norm{x-y}^2)$-smooth. Indeed, by Cauchy-Schwartz inequality and $\eta$-smoothness of $f$, we get
\begin{align*}
|g'(s) - g'(t)|
& = \left| \langle \grad f(y + s(x - y)) - \grad f(y + t(x - y)), x - y \rangle \right| \\
& \le \norm{\grad f(y + s(x - y)) - \grad f(y + t(x - y))} \cdot \norm{x - y} \\
& \le L \norm{(y + s(x - y)) - (y + t(x - y))} \cdot \norm{x - y} \\
& = L \norm{x - y}^2 |s -  t| \; .
\end{align*}
By smoothness of $g$, for any $t \ge 0$,
$$
- \eta \norm{x-y}^2 t \le g'(t) - g'(0) \le \eta \norm{x-y}^2 t
$$
If we integrate both sides:
$$
- \int_0^1 \eta \norm{x-y}^2 t \, dt \le \int_0^1 \left[g'(t) - g'(0)\right] dt \le \int_0^1 \eta \norm{x-y}^2 t \, dt \; ,
$$
we get
$$
- \frac{\eta}{2} \norm{x-y}^2 \le g(1) - g(0) - g'(0) \le \frac{\eta}{2} \norm{x-y}^2 \; .
$$
More compactly,
$$
\left|g(1) - g(0) - g'(0)\right| \le \frac{\eta}{2} \norm{x-y}^2 \; .
$$
This inequality is identical to \eqref{equation:smoothness-proposition-1}.
\end{proof}

\begin{proposition}
If $f:\R^d \to \R$ is a twice continuously differentiable\footnote{A function is twice continuously differentiable if the second derivate exists (at every point) and it is continuous.}
function and
\begin{equation}
\label{equation:smoothness-proposition-2}
\forall x,y \in \R^d \qquad \left| f(x) - f(y) - \langle \grad f(y), x - y \rangle \right| \le \frac{\eta}{2} \norm{x - y}^2 \; .
\end{equation}
then
\begin{equation}
\label{equation:smoothness-proposition-3}
\forall x \in \R^d \qquad \norm{\grad^2 f(x)} \le \eta \; .
\end{equation}
and hence $f$ is $\eta$-smooth.
\end{proposition}

\begin{proof}
We reduce the problem to a one-dimensional one. Fix $x,y \in \R^d$ and define a
scalar function $g:\R \to \R$ as $g(t) = f(y + t (x - y))$. Note that $g(0) =
f(y)$, $g(1) = f(x)$, $g'(t) = \langle \grad f(y + t(x - y)), x - y \rangle$
and $g''(t) = (x-y)^T \grad^2 f(y + t(x - y)) (x - y)$.

We claim that $g$ satisfies analogue of \eqref{equation:smoothness-proposition-2}.
Namely, we claim that
\begin{equation}
\label{equation:smoothness-proposition-for-g}
\forall s,t \in \R \qquad \left| g(s) - g(t) - g'(t) (s - t) \right| \le \frac{\eta}{2} \norm{x - y}^2 (s - t)^2 \; .
\end{equation}
This is easily verified by direct calculation. Indeed,
\begin{align*}
& \left| g(s) - g(t) - g'(t) (s - t) \right| \\
& = \left| f \left( \underbrace{y + s (x - y)}_{=u} \right) - f \left( \underbrace{y + t (x - y)}_{=v} \right) - \left\langle \grad f \left( \underbrace{y + t(x - y)}_{=v} \right), x - y \right\rangle (s - t) \right| \\
& = \left| f(u) - f(v) - \langle \grad f(v), u - v \rangle \right| \\
& \le \frac{\eta}{2} \norm{u - v}^2  \\
& \le \frac{\eta}{2} \norm{x - y}^2 (s - t)^2  \; .
\end{align*}

According to Taylor's theorem, for any $s,t \in \R$, there exists $z$ between $s$ and $t$
such that
\begin{equation}
\label{equation:taylor-theorem}
g(s) = g(t) + g'(t) (s - t) + \frac{1}{2} g''(z) (s - t)^2 \; .
\end{equation}
Combining \eqref{equation:taylor-theorem} and \eqref{equation:smoothness-proposition-for-g},
$$
|g''(z)| = 2 \frac{\left|g(s) - g(t) - g'(t) (s - t)\right|}{(s - t)^2} \le \eta \norm{x - y}^2
$$
Taking limit $s \to t$, using continuity of $g''(z)$, the facts that $z$ lies between $s$ and $t$,
and that $t$ was arbitrary, we obtain
$$
\forall t \in \R \qquad |g''(t)| \le \eta \norm{x - y}^2 \; .
$$
In particular, applying the above inequality for $t = 0$ and substituting for $g''(0)$, we have
$$
|(x-y)^T \grad^2 f(y) (x - y)| \le \eta \norm{x - y}^2 \; .
$$
Since $x,y \in \R^d$ are arbitrary, we have
$$
\forall v,x \in \R^d \qquad |v^T \grad^2 f(x) v| \le \eta \norm{v}^2 \; .
$$
The matrix $\grad^2 f(x)$ is symmetric and thus its spectral norm
is the largest eigenvalue in absolute value. For any eigenvector $v$,
$\|v^T \grad^2 f(x) v\| = \|\lambda\| \norm{v}^2$ where $\lambda$ is the corresponding eigenvalue.
It follows that absolute value of all eigenvalues are bounded by $\eta$.
This proves \eqref{equation:smoothness-proposition-3}.

To prove that $f$ is $\eta$-smooth, we apply Taylor's theorem to
the function $\grad f$. For any $x,y \in \R$ there exists $z \in \R$
on the line segment between $x$ and $y$ such that
$$
\grad f(x) = \grad f(y) + \grad^2 f(z) (x - y) \; .
$$
Threfore, applying \eqref{equation:smoothness-proposition-3} we have
$$
\norm{\grad f(x) - \grad f(y)} = \norm{\grad^2 f(z) (x - y)} \le \norm{\grad^2 f(z)} \norm{x - y} \le \eta \norm{x - y} \; .
$$
and we see that $f$ is $\eta$-smooth.
\end{proof}

We summarize the two propositions.

\begin{corollary}
Let $f:\R^d \to \R$ be a twice continuously differentiable function. The following statements
are equivalent,
\begin{align*}
\forall x,y \in \R^d \qquad & \norm{\grad f(x) - \grad f(x)} \le \eta \norm{x - y} \; , \\
\forall x \in \R^d \qquad & \norm{\grad^2 f(x)} \le \eta \; , \\
\forall x,y \in \R^d \qquad & \left| f(x) - f(y) - \langle \grad f(y), x - y \rangle \right| \le \frac{\eta}{2} \norm{x - y}^2 \; .
\end{align*}

\end{corollary}

\end{document}
